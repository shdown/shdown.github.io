\documentclass[10pt]{article}
\setlength\parindent{0pt}
\usepackage[utf8]{inputenc}
%\usepackage[russian]{babel}

\usepackage{supertabular}
\usepackage{fontawesome}
\usepackage{geometry}
\geometry{
    hmargin=1.5cm,
    vmargin=1.75cm,
    a4paper,
    %showframe,
}
\usepackage[sf,scale=0.95]{libertine}
\usepackage[usenames,svgnames]{xcolor}
\definecolor{text}{HTML}{2b2b2b}
\definecolor{headings}{HTML}{701112}
\definecolor{shade}{HTML}{F5DD9D}
\definecolor{linkcolor}{HTML}{641c1d}
\color{text}

\usepackage{hyperref}
\hypersetup{colorlinks, breaklinks, urlcolor=linkcolor, linkcolor=linkcolor}
\usepackage{fancyhdr}
\pagestyle{fancy}
\fancyhf{}
\renewcommand{\headrulewidth}{0pt}
\usepackage[nobottomtitles*]{titlesec}
\renewcommand{\bottomtitlespace}{0.1\textheight}
\titleformat{\section}{\scshape\LARGE\raggedright}{}{0em}{}[\titlerule]
\titlespacing{\section}{0pt}{0pt}{8pt}

\begin{document}

\parbox[top][0.12\textheight][c]{\linewidth}{
    \vspace{-0.04\textheight}
    \centering
    {\sffamily\Huge Viktor Krapivenskiy}\\\medskip
    {\Huge Curriculum Vitae}
}

\medskip

\section{About}

Computer programmer.
Interested in
  techniques of writing clean and maintainable code,
  software architecture,
  programming language theory,
  compilers,
  parallel programming,
  systems programming,
  Unix philosophy.
Author of a number of side-projects.
Participant of Google Summer of Code---2017.

\medskip

\section{Skills}

Software design $\cdot$
Algorithms and data structures $\cdot$
C (99, 11) $\cdot$
C++ (03, 11, 14, 17) $\cdot$
Python (2 and 3) $\cdot$
Lua $\cdot$
JavaScript $\cdot$
LLVM $\cdot$
POSIX API, Linux API $\cdot$
Crystal $\cdot$
GNU Make, CMake $\cdot$
POSIX sh, bash.

\medskip

%\section{Education}
%
%\begin{tabular}{ l l }
%    2016---present & \textbf{Moscow Institute of Physics and Technology}, Department of Innovations and High Technology \\
%\end{tabular}

\medskip

\section{Experience}

2017 $\cdot$ Summer of Code Intern (Google) $\cdot$ Implemented Lua scripting for the strace project (C, Lua).

\medskip

2018 $\cdot$ Software architect (private company) $\cdot$ Implemented infrastructure, bots and various utilites for analysis of order flow and algorithmic trading on a number of cryptoexchanges (Python, MySQL, Redis);
refactored existing code;
have developed a web interface for accounting and monitoring the status of trading bots in the real time (Python, Flask, JavaScript);
have developed a high-performance log analyzer (C++).

\medskip

2019 $\cdot$ C++ developer (\href{https://offscale.io}{Offscale}) $\cdot$ Developed \href{https://github.com/offscale/liboffkv}{liboffkv}, a uniform interface for distributed key-value storages, in a team of four; implemented C bindings; made a contribution to \href{https://github.com/oliora/ppconsul}{ppconsul}: transactions support (C++).

\medskip

2019 $\cdot$ Software architect (contract with \href{https://research.sikoba.com}{Sikoba Research}) $\cdot$ Implemented support for LLVM in the verifiable computation framework \href{https://github.com/sikoba/isekai}{isekai} (Crystal).
See the following articles for more information:
\begin{itemize}
    \item \href{https://medium.com/sikoba-network/isekai-technical-update-llvm-d5003fc8f009}{Isekai LLVM update \#1};
    \item \href{https://medium.com/sikoba-network/isekai-llvm-update-2-conditionals-and-loops-81296a0eccbf}{Isekai LLVM update \#2: conditionals and loops};
    \item \href{https://medium.com/sikoba-network/isekai-llvm-final-update-894fb6863fcf}{Isekai LLVM: final update}.
\end{itemize}

\medskip

\section{Awards}

\begin{tabular}{ l l }
    2016 & Prizewinner of the All-Russian Olympiad in Informatics, Finals \\
    2016 & Gold winner of the Individual Olympiad of School Students in Informatics and Programming, Finals \\
    2017 & 4\textsuperscript{th} place in ``LAToken hackathon'': smart contract for tokenization of different kinds of assets \\
    2018 & 1\textsuperscript{st} place in ``Global Changers'' hackathon: client support bot system \\
    2018 & 1\textsuperscript{st} place in ``IDACB \& CryptoBazar hackathon'': chat based on proxy re-encyption protocol \\
    2018 & 1\textsuperscript{st} place in ``Phystech.Genesis'' hackathon: mobile application for traveling \\
    2018 & 3\textsuperscript{st} place in ``CryptoBazar Serial Hacking: October'': PoC software raytracer using Intel SGX \\
    2018 & 1\textsuperscript{st} place in ``CryptoBazar Serial Hacking: November'': LLVM IR interpreter with register-based VM \\
    2018 & Mentorship of two teams at ``CryptoBazar Serial Hacking: December'' that took 2\textsuperscript{nd}---3\textsuperscript{rd} places \\
    2019 & 1\textsuperscript{st} place in ``CryptoBazar Serial Hacking: Grand Finale'': network traffic record/replay tool \\
\end{tabular}

\medskip

\section{Projects}

\begin{tabular}{ l l }
    2016---present & \href{https://github.com/shdown/luastatus}{\textbf{luastatus}}, a universal status bar content generator \\
    2017 & \href{http://0x1.tv/img_auth.php/f/fe/Lua-%D1%81%D0%BA%D1%80%D0%B8%D0%BF%D1%82%D0%B8%D0%BD%D0%B3_%D0%B2_strace_%28%D0%92%D0%B8%D0%BA%D1%82%D0%BE%D1%80_%D0%9A%D1%80%D0%B0%D0%BF%D0%B8%D0%B2%D0%B5%D0%BD%D1%81%D0%BA%D0%B8%D0%B9%2C_OSSDEVCONF-2017%29.pdf}{\textbf{support for Lua scripting in strace}}, Google Summer of Code---2017 project \\
    2018---present & \href{https://github.com/shdown/calx}{\textbf{calx}}, a bc-like programming language \\
    2018 & \href{https://github.com/angou-exchange-utils}{\textbf{angou}}, connectors for a number of cryptoexchanges \\
    2019 & \href{https://github.com/shdown/stalk-tool}{\textbf{stalk-tool}}, a VK app to search for comments of a specified user \\
\end{tabular}

\medskip

\section{References}

\parbox[top][0.12\textheight][c]{\linewidth}{
    \vspace{-0.04\textheight}
    \colorbox{shade}{
        \begin{supertabular}{p{0.05\linewidth}|p{0.775\linewidth}}
            %\raisebox{0pt}{\small  \faEnvelope}        & \href{mailto:krapivenskiy.va@phystech.edu}{krapivenskiy.va@phystech.edu} \\
            \raisebox{0pt}{\small  \faEnvelope}        & \href{mailto:shdownnine@gmail.com}{shdownnine@gmail.com} \\
            \raisebox{-1pt}{\small \faGithubAlt}       & \href{https://github.com/shdown}{https://github.com/shdown} \\
            \raisebox{-1pt}{\small \faLinkedinSquare}  & \href{https://www.linkedin.com/in/victor-krapivensky-297912146/}{https://www.linkedin.com/in/victor-krapivensky-297912146/} \\
            %\raisebox{-1pt}{\small \faMapSigns}        & Dolgoprudny, Moscow oblast, Russia \\
        \end{supertabular}
    }
}

\end{document}
