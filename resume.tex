\documentclass[10pt]{article}
\setlength\parindent{0pt}
\usepackage[utf8]{inputenc}
%\usepackage[russian]{babel}

\usepackage{supertabular}
\usepackage{fontawesome}
\usepackage{geometry}
\usepackage{tabularx}
\geometry{
    hmargin=1.5cm,
    vmargin=1.75cm,
    a4paper,
    %showframe,
}
\usepackage[sf,scale=0.95]{libertine}
\usepackage[usenames,svgnames]{xcolor}

\definecolor{text}{HTML}{2b2b2b}
\definecolor{headings}{HTML}{701112}
\definecolor{shade}{HTML}{F5DD9D}
\definecolor{linkcolor}{HTML}{641c1d}
\color{text}

\usepackage{hyperref}
\hypersetup{colorlinks, breaklinks, urlcolor=linkcolor, linkcolor=linkcolor}
\usepackage{fancyhdr}
\pagestyle{fancy}
\fancyhf{}
\renewcommand{\headrulewidth}{0pt}
\usepackage[nobottomtitles*]{titlesec}
\renewcommand{\bottomtitlespace}{0.1\textheight}
\titleformat{\section}{\scshape\LARGE\raggedright}{}{0em}{}[\titlerule]
\titlespacing{\section}{0pt}{0pt}{8pt}

\begin{document}

\parbox[top][0.12\textheight][c]{\linewidth}{
    \vspace{-0.04\textheight}
    \centering
    {\sffamily\Huge Viktor Krapivenskiy}\\\medskip
    {\Huge Curriculum Vitae}
}

\medskip

\section{About}

Regarded by many as a computer programmer.
Interested in
  techniques of writing clean and maintainable code,
  software design,
  compilers,
  parallel programming,
  systems programming.
Author of a number of side projects.
Participant of Google Summer of Code---2017.

\medskip

\section{Skills}

Software design $\cdot$
Algorithms and data structures $\cdot$
C $\cdot$
C++ $\cdot$
POSIX API, Linux API $\cdot$
LLVM $\cdot$
Go $\cdot$
Python $\cdot$
Lua $\cdot$
JavaScript $\cdot$
x86-64 assembly.

\medskip

%\section{Education}
%
%\begin{tabular}{ l l }
%    2016---present & \textbf{Moscow Institute of Physics and Technology}, Department of Innovations and High Technology \\
%\end{tabular}
%
%\medskip

\section{Experience}

2021---present $\cdot$ Software architect (private company) $\cdot$ Developed market data providers for multiple exchanges, programs to perform algorithmic trading on multiple exchanges, programs for low-latency transmission of market data over the network, and other tools, in C $\cdot$ Implemented a fast JSON parser in C $\cdot$ Implemented efficient parallel calculation of a digital signature based on Pedersen hash, needed for dYdX cryptocurrency exchange, in x86-64 assembly and C $\cdot$ Implemented a fast emulator of EVM programs to calculate price slippage for a given amount for SushiSwap, Uniswap v2 and v3 pools, in x86-64 assembly and C.

\medskip

2020 $\cdot$ Go developer (contract with \href{https://offscale.io}{Offscale}) $\cdot$ Developed \href{https://github.com/offscale/goffkv}{goffkv} (\href{https://github.com/offscale/goffkv-consul}{goffkv-consul}, \href{https://github.com/offscale/goffkv-zk}{goffkv-zk}, \href{https://github.com/offscale/goffkv-etcd}{goffkv-etcd}) --- a rewrite of \href{https://github.com/offscale/liboffkv}{liboffkv} in Go.

\medskip

2019 $\cdot$ Software developer (contract with \href{https://fantom.foundation}{Fantom foundation}) $\cdot$ Developed tools for internal use.

\medskip

2019 $\cdot$ Software architect (contract with \href{https://research.sikoba.com}{Sikoba Research}) $\cdot$ Implemented support for LLVM in the verifiable computation framework \href{https://github.com/sikoba/isekai}{isekai} (Crystal). See the following articles for more information:
\begin{itemize}
    \item \href{https://medium.com/sikoba-network/isekai-technical-update-llvm-d5003fc8f009}{Isekai LLVM update \#1};
    \item \href{https://medium.com/sikoba-network/isekai-llvm-update-2-conditionals-and-loops-81296a0eccbf}{Isekai LLVM update \#2: conditionals and loops};
    \item \href{https://medium.com/sikoba-network/isekai-llvm-final-update-894fb6863fcf}{Isekai LLVM: final update}.
\end{itemize}

\medskip

2019 $\cdot$ C++ developer (contract with \href{https://offscale.io}{Offscale}) $\cdot$ Developed \href{https://github.com/offscale/liboffkv}{liboffkv}, a uniform interface for distributed key-value storages, in a team of four; implemented C bindings; made a contribution to \href{https://github.com/oliora/ppconsul}{ppconsul}: transactions support (C++).

\medskip

2018 $\cdot$ Software architect (private company) $\cdot$ Implemented bots and various utilites for analysis of order flow and trading on a number of cryptoexchanges (Python, MySQL).

\medskip

2017 $\cdot$ Summer of Code Intern (Google) $\cdot$ Implemented Lua scripting for the strace project (C, Lua).

\medskip

\section{Awards}

\begin{tabular}{ l l }
    2016 & Prizewinner of the All-Russian Olympiad in Informatics, Finals \\
    2016 & Gold winner of the Individual Olympiad of School Students in Informatics and Programming, Finals \\
    2017 & 4\textsuperscript{th} place in ``LAToken hackathon'': smart contract for tokenization of different kinds of assets \\
    2018 & 1\textsuperscript{st} place in ``Global Changers'' hackathon: client support bot system \\
    2018 & 1\textsuperscript{st} place in ``IDACB \& CryptoBazar hackathon'': chat based on proxy re-encyption protocol \\
    2018 & 1\textsuperscript{st} place in ``Phystech.Genesis'' hackathon: mobile application for traveling \\
    2018 & 3\textsuperscript{rd} place in ``CryptoBazar Serial Hacking: October'': PoC software raytracer using Intel SGX \\
    2018 & 1\textsuperscript{st} place in ``CryptoBazar Serial Hacking: November'': LLVM IR interpreter with register-based VM \\
    2018 & Mentorship of two teams at ``CryptoBazar Serial Hacking: December'' that took 2\textsuperscript{nd}---3\textsuperscript{rd} places \\
    2019 & 1\textsuperscript{st} place in ``CryptoBazar Serial Hacking: Grand Finale'': network traffic record/replay tool \\
    2020 & 2\textsuperscript{nd} place in ``VirusHack'': automatic detection of deviations in a video stream \\
\end{tabular}

\medskip

\section{Projects}

\begin{tabularx}{\textwidth}{ l X }

    2016---present & \href{https://github.com/shdown/luastatus}{\textbf{luastatus}}, a universal status bar content generator
    \newline
    \footnotesize{
        luastatus is a universal status bar content generator. It allows the user to configure the way the data from event sources is processed and shown, with Lua.
        Its main feature is that the content can be updated immediately as some event occurs, be it a change of keyboard layout, active window title, volume or a song in your favorite music player (provided that there is a plugin for it) — a thing rather uncommon for tiling window managers.
        Its motto is:
        \begin{quote}
        No more heavy-forking, second-lagging shell-script status bar generators!
        \end{quote}
        It has a modular architecture, supporing plugins for providing data and barlibs for interacting with different status bars.
        It supports \textit{i3wm}, \textit{dwm}, \textit{lemonbar}, \textit{dzen}/\textit{dzen2}, \textit{xmobar}, \textit{yabar}, \textit{dvtm}, and others.
    }
    \\
    \medskip

    2017 & \href{http://0x1.tv/img_auth.php/f/fe/Lua-\%D1\%81\%D0\%BA\%D1\%80\%D0\%B8\%D0\%BF\%D1\%82\%D0\%B8\%D0\%BD\%D0\%B3_\%D0\%B2_strace_\%28\%D0\%92\%D0\%B8\%D0\%BA\%D1\%82\%D0\%BE\%D1\%80_\%D0\%9A\%D1\%80\%D0\%B0\%D0\%BF\%D0\%B8\%D0\%B2\%D0\%B5\%D0\%BD\%D1\%81\%D0\%BA\%D0\%B8\%D0\%B9\%2C_OSSDEVCONF-2017\%29.pdf}{\textbf{support for Lua scripting in strace}}, Google Summer of Code---2017 project
    \newline
    \footnotesize{
        I extended the \textit{strace} project with tampering capability, allowing the user to inject fake syscall results, and read and write the memory of the process being traced.
    }
    \\
    \medskip

    2020 & \href{https://github.com/shdown/libdeci}{\textbf{libdeci}}, an arbitrary-precision decimal arithmetic library for C
    \newline
    \footnotesize{
        This is an arbitrary-precision decimal arithmetic library for C with add-on libraries
        \href{https://github.com/shdown/libdeci-kara}{\textbf{libdeci-kara}} implementing Karatsuba multiplication,
        \href{https://github.com/shdown/libdeci-ntt}{\textbf{libdeci-ntt}} implementing multiplication via Number-Theoretic Transform (NTT), a variant of Fourier transform,
        and \href{https://github.com/shdown/libdeci-newt}{\textbf{libdeci-newt}} implementing fast inversion and division using Newton's method.
        It is faster than the \textit{mpdecimal} library.
    }
    \\
    \medskip

    2020---present & \href{https://github.com/shdown/calx}{\textbf{calx}}, a bc-like programming language
    \newline
    \footnotesize{
        \textit{calx} an attempt to make a modern replacement for bc, while preserving its best features, such as big-decimal numbers and explicit support for interactivity in the language.
        It is a full-fledged programming language with functions, local and global variables, lists, dicts, strings, etc.
    }
    \\
    \medskip

    2020 & \href{https://github.com/shdown/decimal-multiplication-paper}{\textbf{``Speeding up decimal multiplication''}}, a research project (\href{https://arxiv.org/abs/2011.11524}{\textbf{ArXiv.org URL}}).
    \newline
    \footnotesize{
        This research project achieves a 3x---5x speedup over the \textit{mpdecimal} library.
        The paper describes the implementation and discuss further possible optimizations. It also present a simple cache-efficient algorithm for in-place
        $2^n \times n$ or $n \times 2^n$ matrix transposition, the need for which arises in the
        ``six-step algorithm'' variation of the matrix Fourier algorithm, and
        which does not seem to be widely known. Another finding is that use
        of two prime moduli instead of three makes sense even considering the
        worst case of increasing the size of the input, and makes for simpler
        answer recovery.
    }
    \\
    \medskip

    2022 & \href{https://github.com/shdown/fiwia}{\textbf{FiWiA}}, a generator of x86-64 machine code for fixed-width multi-word arithmetics
    \newline
    \footnotesize{
        The need for fixed-width multi-word arithmetics frequently arises in cryptography.
        In this setting, full unroll is usually desirable in two reasons: loop overhead and the need to pass the carry flag to the next iteration,
        which, without unrolling, would have to be done via saving the carry in a register and restoring it in the next iteration,
        which is suboptimal in performance.
        \textit{fiwia} generates fully unrolled x86-64 assembly for fixed-width arithmetic operations, such as:
        addition/subtraction,
        masked addition/subtraction,
        negation,
        comparison,
        multiplication,
        bit shifts.
    }
    \\
    \medskip

    2022 & \href{https://github.com/shdown/sloppy-json}{\textbf{sloppy-json}}, span-oriented C JSON parser with an external preprocessor to sweeten the process of parsing
    \newline
    \footnotesize{
        Ergonomic parsing of JSON in C is quite a challenging task: virtually all existing parsers are either:
        \begin{itemize}
            \item DOM-like (e.g. \textit{cJSON}): they convert JSON to ``objects'' with \textbf{dynamically allocated} arrays and dictionaries, which the user is supposed to fetch data from and then destroy the ``objects''; or
            \item SAX-like (e.g. \textit{YAJL}): they allow the user to iterate over JSON, which in practice means a lot of callbacks, which are tedious to write and lead to sufficient performance loss, because the callback functions are called by pointer and cannot be inlined.
        \end{itemize}
        In settings where we cannot afford dynamic memory allocation, e.g. when tail latency is important, DOM-like parsers cannot be used.
        Instead of focusing on JSON objects, \textit{sloppy-json} instead operates on spans (a span is a pair of pointer and length) which, when parsed, represent JSON objects.
        It provides functions to classify a span, iterate over spans representing arrays and dictionaries (the iterator is a span or a pair of spans), convert spans into numbers, unescaping string spans, and everything else needed to ``parse'' JSON.
        It does not allocate any memory.
        It also comes with an external preprocessor written in python to ``sweeten'' the process of parsing.
        It also provides some utility functions to generate JSON (e.g. formatting numbers and escaping strings).
    }
    \\
\end{tabularx}

\medskip

\section{References}

\parbox[top][0.12\textheight][c]{\linewidth}{
    \vspace{-0.04\textheight}
    \colorbox{shade}{
        \begin{supertabular}{p{0.05\linewidth}|p{0.775\linewidth}}
            %\raisebox{0pt}{\small  \faEnvelope}        & \href{mailto:krapivenskiy.va@phystech.edu}{krapivenskiy.va@phystech.edu} \\
            \raisebox{0pt}{\small  \faEnvelope}        & \href{mailto:shdownnine@gmail.com}{shdownnine@gmail.com} \\
            \raisebox{-1pt}{\small \faGithubAlt}       & \href{https://github.com/shdown}{https://github.com/shdown} \\
            \raisebox{-1pt}{\small \faLinkedinSquare}  & \href{https://www.linkedin.com/in/shdownnine}{https://www.linkedin.com/in/shdownnine} \\
            %\raisebox{-1pt}{\small \faMapSigns}        & Dolgoprudny, Moscow oblast, Russia \\
        \end{supertabular}
    }
}

\end{document}
